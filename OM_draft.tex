%% LyX 2.2.1 created this file.  For more info, see http://www.lyx.org/.
%% Do not edit unless you really know what you are doing.
\documentclass[12pt,letterpaper]{extarticle}
\usepackage[T1]{fontenc}
\usepackage[latin9]{inputenc}
\setcounter{secnumdepth}{2}
\setcounter{tocdepth}{2}
\usepackage[active]{srcltx}
\usepackage{amsmath}
\usepackage{amsthm}
\usepackage{amssymb}
\usepackage{setspace}
\usepackage[authoryear]{natbib}
\doublespacing

\makeatletter

%%%%%%%%%%%%%%%%%%%%%%%%%%%%%% LyX specific LaTeX commands.
\special{papersize=\the\paperwidth,\the\paperheight}


%%%%%%%%%%%%%%%%%%%%%%%%%%%%%% Textclass specific LaTeX commands.
  \theoremstyle{definition}
  \newtheorem{defn}{\protect\definitionname}
\theoremstyle{plain}
\newtheorem{thm}{\protect\theoremname}
  \theoremstyle{plain}
  \newtheorem{lem}{\protect\lemmaname}
  \theoremstyle{plain}
  \newtheorem{prop}{\protect\propositionname}

%%%%%%%%%%%%%%%%%%%%%%%%%%%%%% User specified LaTeX commands.
%&latex

\linespread{1.05}



\usepackage{amsthm}
\usepackage[bottom]{footmisc}

\usepackage{pdfsync}

\usepackage{natbib}
\setlength{\bibsep}{0.0pt}

\makeatother

  \providecommand{\definitionname}{Definition}
  \providecommand{\lemmaname}{Lemma}
  \providecommand{\propositionname}{Proposition}
\providecommand{\theoremname}{Theorem}

\begin{document}

\title{(Non-)Obvious Manipulability (???)}

\author{Thayer Morrill \\Department of Economics \\North Carolina State University \and Peter Troyan\\ Department of Economics \\ University of Virginia
}
\maketitle
\begin{abstract}
Abstract TBD
\end{abstract}

\section{Introduction}

This is the greatest paper ever. 

\subsection{Motivating example?}

Maybe include a motivating example (e.g., school choice, where Boston
is obviously manipulable, but EADA is not)

\section{Model}

There is a finite set of agents $I$ and a finite set of outcomes
$X$. The set $\Theta_{i}$ denotes the set of possible \textbf{types
}for agent $i$, with generic element denoted $\theta_{i}\in\Theta_{i}$.
Each agent has a utility function $u_{i}:X\times\Theta_{i}\rightarrow\mathbb{R}$.
For some of our applications, such as school choice, only the ordinal
information contained in the utility function will be relevant, and
$\theta_{i}$ can be thought of as encoding an agent's ordinal preference
ranking over alternatives. For other applications, such as auctions,
agent utilities will be quasilinear in money.  

Let $\Theta=\Pi_{i\in I}\Theta_{i}$. A \textbf{mechanism }is a function
$f:\Theta\rightarrow X$ that maps type profiles to outcomes. The
standard definition of strategyproofness requires that truthful reporting
be a weakly dominant strategy in the preference revelation game induced
by the mechanism, i.e., for all $i$, $u_{i}(f(\theta_{i},\theta_{-i}),\theta_{i})\geq u_{i}(f(\theta_{i}',\theta_{-i}),\theta_{i})$
for all $i,$ all $\theta_{i},\theta_{i}'\in\Theta_{i}$, and all
$\theta_{-i}\in\Theta_{-i}$. If there exists some $\theta_{i},\theta_{i}'$
and $\theta_{-i}$ such that $u_{i}(f(\theta_{i}',\theta_{-i}),\theta_{i})>u_{i}(f(\theta_{i},\theta_{-i}),\theta_{i})$
then $f$ is said to be \textbf{manipulable}, and reporting $\theta_{i}'$
is a \textbf{manipulation} for type $\theta_{i}$. Note that for a
mechanism to be classified as manipulable, there must simply be \emph{some
}profile of for the other agents $\theta_{-i}$ such that reporting
$\theta_{i}'$ is better than reporting the true $\theta_{i}$ for
agent $i$. However, in other instances, reporting $\theta_{i}'$
may actually be worse for agent $i$ than reporting truthfully. Thus,
it may be very non-obvious whether such a manipulation will be profitable.
Additionally, determining all possible outcomes from a potential report
$\theta_{i}'$ can be a very difficult and time-consuming process;
in many cases, agents may more naturally just look for the best and/or
worst outcomes from two potential reports, and compare those \textbf{{[}probably
should justify this better, maybe add some references?{]}}. These
ideas motivate the following definition.
\begin{defn}
Consider an agent $i$ of type $\theta_{i}$. Reporting $\theta_{i}'\neq\theta_{i}$
is an \textbf{obvious manipulation }of mechanism $f$ if \medskip

(i) $\inf_{\theta_{-i}}u_{i}(f(\theta_{i}',\theta_{-i}),\theta_{i})>\inf_{\theta_{-i}}u_{i}(f(\theta_{i},\theta_{-i}),\theta_{i})$,
or \medskip

(ii) $\inf_{\theta_{-i}}u_{i}(f(\theta_{i}',\theta_{-i}),\theta_{i})=\inf_{\theta_{-i}}u_{i}(f(\theta_{i},\theta_{-i}),\theta_{i})$
and $\sup_{\theta_{-i}}u_{i}(f(\theta_{i}',\theta_{-i}),\theta_{i})>\sup_{\theta_{-i}}u_{i}(f(\theta_{i},\theta_{-i}),\theta_{i})$. 
\end{defn}
In words, an obvious manipulation is a manipulation such that lying
either (i) makes the agent better off in the worst-case or (ii) makes
the agent better off in the best-case, without harming her in the
worst case. If neither (i) nor (ii) hold, then reporting $\theta_{i}'\neq\theta_{i}$
is a \textbf{non-obvious manipulation}.\footnote{In particular, this holds if lying makes the agent strictly worse
off in the worst case, i.e., $\inf_{\theta_{-i}}u_{i}(f(\theta_{i}',\theta_{-i}),\theta_{i})<\inf_{\theta_{-i}}u_{i}(f(\theta_{i},\theta_{-i}),\theta_{i})$.}

\section{School choice}

We consider a canonical model of school choice, as in the seminal
paper of \citet{Abdulkadiroglu:AER:2003}. Let $S$ be a set of schools.
An outcome $x\in X$ is then a mapping $x:I\rightarrow S$ that returns
each student's assigned school. A student's utility from an outcome
$x$ depends only on her own assignment, and so for this section,
an agent's type $\theta_{i}$ can be identified with a vector $\theta_{i}\in\mathbb{R}^{|S|}$,
and we can define utility functions over schools $a\in S$ as $u_{i}(a,\theta_{i})=\theta_{i}^{a}.$

Each school $a$ has a\textbf{ }priority relation $\succ_{a}$ over
the set of students (agents) $I$. An additional important desiderata
in school choice is \textbf{fairness }(sometimes called \textbf{no
justified envy}). Given an outcome $x$, we say that student $i$
\textbf{claims a seat }at school $a$ if $u_{i}(a,\theta_{i})>u_{i}(x(i),\theta_{i})$
and either (i) school $a$ has an empty seat or (ii) there exists
a student $j$ such that $x(j)=a$ and $i\succ_{a}j$. If no student
claims any seat, then outcome $x$ is said to be \textbf{fair}.\textbf{ }

One of the most popular mechanisms in school choice is the deferred
acceptance (DA) mechanism of \citet{Gale:AMM:1962}. Two important
properties of DA are that it is strategyproof and fair; the main drawback
is that it is not efficient. Because it is not efficient, it is important
to understand whether we can improve upon DA from a welfare perspective.
Formally, given two mechanisms $f,g$, say that $f$ \textbf{Pareto
dominates }$g$ if $u_{i}(f(\theta),\theta_{i})\geq u_{i}(g(\theta),\theta_{i})$
for all $i$ and all $\theta\in\Theta$, with the inequality strict
for at least one $i,\theta$. \citet{Abdulkadiroglu:AER:2009} and
\citet{kesten:QJE:2010} show that any mechanism that Pareto dominates
DA is necessarily manipulable . However, as our main result shows,
while such mechanisms may be manipulable, they are not obviously manipulable. 
\begin{thm}
Let $f^{DA}$ be the $DA$ mechanism, and $g$ be a mechanism that
Pareto dominates $f^{DA}$. Then, $g$ is not obviously manipulable. 
\end{thm}
\begin{proof}
Given a student $i$ of type $\theta_{i}$, define the \textbf{upper
contour set at $a$ }as $\pi_{i}(a)=\{b\in S:u_{i}(b,\theta_{i})\geq u_{i}(a,\theta_{i})\}$.
Any subset $S'\subset\pi_{i}(a)$ is referred to as an \textbf{upper
contour subset at $a$}. Let $B_{i}(a)=\{j\in I:j\succ_{a}i\}$ be
the set of students who have higher priority at $a$ than $i$. Last,
given a mechanism $f$, let 
\[
\omega_{i}^{f}(\theta_{i})=\min_{\theta_{-i}}u_{i}(f(\theta_{i},\theta_{-i}),\theta_{i}).
\]
 In words, $\omega_{i}^{f}(\theta_{i})$ is the worst-case outcome
for $i$ in mechanism $f$ when she has preferences $\theta_{i}$,
taken over all reports of the other agents.
\begin{lem}
For any school $a\in S$, $u_{i}(a,\theta_{i})>\omega_{i}^{f}(\theta_{i})$
if and only if, for every upper contour subset $S'\subset\pi_{i}(a)$,
the following holds:
\[
\sum_{s\in S'}q_{s}\leq|\cup_{s\in S'}B_{i}(s)|.
\]
\end{lem}
\emph{Proof of lemma. $\square$}
\end{proof}
To show that a mechanism is obviously manipulable, it suffices to
exhibit a student who has an obvious manipulation. For the Boston
mechanism, this is provided in the motivating example discussed in
the introduction. For completeness, we state this as the following
result.
\begin{prop}
The Boston mechanism is obviously manipulable. 
\end{prop}

\section{Other applications}

\subsection{Two-sided matching}

\subsection{Auctions}

\subsection{Others?}

\section{Conclusion}

Wasn't this the greatest paper ever?

\bibliographystyle{ecta}
\bibliography{\string"/Users/pgt8y/Dropbox/Research/master-bib-file\string"}

\end{document}
